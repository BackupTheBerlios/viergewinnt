\documentclass{article}
\usepackage[latin1]{inputenc}
\usepackage{ngerman}
\usepackage{a4wide}

\setlength\parskip{\medskipamount}
\setlength\parindent{0pt}


\begin{document}
\title{\begin{Huge}Pflichtenheft\end{Huge}\\ \vspace{1em} Projekt "`Vier Gewinnt"'}
\author{Manuel Carrara\\ Katharina Grimme\\ Philipp Herk\\ Malte Kn�rr}
\date{}
\maketitle
%\vfill{}

\vspace{0,3cm}
{\centering \begin{tabular}{|c|c|c|}
\hline
Version&
Datum&
Autor\\
\hline
\hline
1.0&
02.12.2002&
Katharina Grimme\\
\hline
0.1&
29.11.2002&
Malte Kn�rr\\
\hline
\end{tabular}\par}
\vspace{0,3cm}

%\newpage

%\tableofcontents{} \newpage

\section{Einleitung}
Als diese Version des Pflichtenhefts geschrieben wurde, befand sich das Projekt in der Anfangsphase. Die Anforderungen wurden aus der Aufgabenstellung von Prof. Dr. Stahl �bernommen (ab und zu auf �nderungen �berpr�fen).

\section{Produktfunktionen}
\begin{quote}
Das Ziel dieses zwei Personen Spiels ist es, vier gleichfarbige Steine vertikal, horizontal oder diagonal anzuordenen.
\end{quote}

Welche der optionalen Funktionen entwickeln wir? Planen wir das erst nach der Blockwoche oder schon vorher?

\subsection{Grundlegende Funktionalit�t} 
\begin{quote}
Grundlegende Funktionalit�t, Implementierung des Minimax Algorithmus. Sowohl der Computer als auch der menschliche Spieler sollen den ersten Zug machen k�nnen. Variable Spielst�rke. (max. 20 Punkte)
\end{quote} 

\subsection{UI}
\begin{quote}
Textbasierte und graphische Benutzerschnittstelle. Dokumentiertes API so dass auch zwei Computer gegeneinander spielen k�nnen. (max. 20 Punkte)
\end{quote}

\subsection{Alpha-Beta Pruning}
\begin{quote}
Effizienz durch Alpha-Beta Pruning. (max. 20 Punkte)
\end{quote}

\subsection{Heuristiken}
\begin{quote}
Heuristiken f�r eine gute Stellungsbewertung. (max. 20 Punkte)
\end{quote}

\subsection{Lernen (optional)}
\begin{quote}
Aus gespielten Partien lernen und sich dadurch selbst�ndig verbessern. (max. 30 Punkte)
\end{quote}

\subsection{Nebenl�ufigkeit (optional)}
\begin{quote}
Die Zeit ausnutzen, in der der Gegner nachdenkt. (max. 30 Punkte)
\end{quote}

\subsection{Ergebnis-Wiederverwenung (optional)}
\begin{quote}
Teilergebnisse zur Berechnung eines Zuges im n�chsten Zug wiederverwenden. (max. 20 Punkte)
\end{quote}

\subsection{Netz (optional)}
\begin{quote}
�ber's Netz mit dem Programm einer anderen Gruppe spielen, Client Server Architektur. (max. 20 Punkte)
\end{quote}
Was meint "`Client-Server-Architektur"' hier genau?

\section{Risiken}
\subsection{Kooperation mit anderen Gruppen}
F�r die Funktion "`Netz"' muss mit den anderen Gruppen eine Schnittstelle vereinbart werden. Ansonsten besteht die Gefahr, dass wir aneinander vorbei entwickeln.

\section{Nicht-funktionale Anforderungen}
\subsection{Dokumentation}
\begin{quote}
Dokumentation des Quelltextes, der verwendeten Algorithmen und Heuristiken. (max. 20 Punkte) 
[...] In der Dokumentation muss stehen welche Kriterien das Programm erf�llt und welche Klassen hierf�r zust�ndig sind.
\end{quote}

\end{document}
